%%%%%%%%%%%%%%%%%%%%%%%%%%%%%%%%%%%%%%%%%%%%%%%%%%%%%%%%%%%%%%%%%%%%%%%%%%%%%%%%
\intro
%%%%%%%%%%%%%%%%%%%%%%%%%%%%%%%%%%%%%%%%%%%%%%%%%%%%%%%%%%%%%%%%%%%%%%%%%%%%%%%%

В современном постиндустриальном обществе наблюдаются тенденции к постепенному переходу от традиционных форм обучения к индивидуальным образовательным траекториям. Такой подход повышает эффективность обучения засчет возможности выбора оптимальных форм, средств, методов и сроков обучения, которые в наибольшей степени подходят под индивидуальные особенности обучающегося, способствуя формированию профессионального подхода обучающегося к его будущей деятельности и личной заинтересованности в изучении выбранных курсов \cite{Furin}.

Одним из ключевых инструментов реализации индивидуальных траекторий обучения являются образовательные платформы. Стоит отметить, что платформы могут также использоваться для дистанционного обучения в классическом формате академических групп.

Важность готовности к частичному или полному переносу обучения в онлайн также демонстрирует текущая эпидемиологическая обстановка в мире: обладая необходимыми технологиями, образовательные учереждения могут дать возможность обучающимся получить заявленные в образовательных программах компетенции при наличии у последних всего лишь персональных компьютеров с доступом к сети Интернет.

Одной из подобных технологий является образовательная платформа openEDX, позволяющая как запускать курсы на портале edx.org, так и построить собственную образовательную систему. С целью повышения вовлеченности обучающихся в образовательный процесс openEDX использует технологию XBlock, позволяющую встраивать интерактивные компоненты в разделы курса.

Таким образом, в связи с разработкой образовательной платформы на базе openEDX для обучения робототехнике детей 5-14 лет, целью данной работы была поставлена разработка компонента для образовательной платформы на базе openEDX, позволяющего обучающимся составлять программу для перемещения робота в двумерной плоскости и симулирующего выполнение этой программы.